%---------------------------------------------------------------------
%
%                          Cap�tulo 3
%
%---------------------------------------------------------------------

\chapter{Herramientas utilizadas}

\begin{FraseCelebre}
	\begin{Frase}
		" La fuerza no viene de la capacidad corporal, sino de la voluntad del alma". 
	\end{Frase}
	\begin{Fuente}
		Mahatma Gandhi
	\end{Fuente}
\end{FraseCelebre}


Introducci�n capitulo 3 COMPLETAR 
\\



%-------------------------------------------------------------------
\section{Flask}
%-------------------------------------------------------------------
\label{cap2:sec:Flask}
La discapacidad auditiva es la dificultad que sufren algunas personas para percibir el sonido a trav�s del o�do. El colectivo de personas con d�ficit auditivo es muy heterog�neo, influyendo factores como la edad en la que aparece la sordera, el grado de �sta, as� como factores de su entorno educativo y social. Eexisten distintas clasificaciones \citep*{tchDiscapnet}:\\

\newpage
\begin{enumerate}
	
	\item Seg�n el grado de p�rdida de audici�n:
	
	\begin{itemize}
		\item \textbf{Audici�n normal:} se perciben sonidos por encima de 20 decibelios.
		\item \textbf{Hipoacusia:}  p�rdida parcial de la audici�n.
		
		\begin{itemize}
			\item \textit{\textbf{Leve:}} no se perciben sonidos inferiores a 40 decibelios.
			\item \textit{\textbf{Moderada:}} se presentan p�rdidas entre 40 y 70 decibelios
			\item \textit{\textbf{Severa:}} p�rdida de entre 70 y 90 decibelios. En este grado se requiere de uso de ayudas auditivas, como pr�tesis o implantes. A partir de p�rdidas de 75 decibelios la Seguridad Social considera al individuo persona sorda.
		\end{itemize}
		
		\item \textbf{Sordera (Cofosis):} p�rdida total de la audici�n. Se precisa de la ayuda de c�digos de comunicaci�n alternativa.
	\end{itemize}
	
	\item Seg�n su etiolog�a: 
	
	\begin{itemize}
		\item \textbf{Gen�ticas:} factores hereditarios influyen en la p�rdida de audici�n del individuo.	
		\item \textbf{Adquiridas:} influyen factores externos como golpes o exposici�n a ruidos fuertes.
		\item \textbf{Cong�nitas:} la p�rdida de audici�n est� presente desde el nacimiento del individuo.
		
	\end{itemize}
	
	\item Seg�n el momento de la aparici�n se distinguen: 
	
	\begin{itemize}
		\item \textbf{Prelocutivas:} la sordera aparece antes de que el individuo haya aprendido el lenguaje oral. La gran mayor�a de este colectivo no sabe ni leer ni escribir.
		
		\item \textbf{Postlocutivas:}  la persona es capaz de comunicarse oralmente antes de la aparici�n de la discapacidad. En condiciones normales conoce el lenguaje escrito.
		
	\end{itemize}
\end{enumerate}

Muchas personas con discapacidad auditiva necesitan apoyar la comunicaci�n oral con Sistemas Aumentativos y Alternativos de Comunicaci�n (SAAC). En la siguiente secci�n se profundizar� en estos m�todos de comunicaci�n y se explicar�n con detalle los tipos que m�s se utilizan. \\















