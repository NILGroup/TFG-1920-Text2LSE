%---------------------------------------------------------------------
%
%                          Cap�tulo 3
%
%---------------------------------------------------------------------

\chapter{Herramientas utilizadas}

\begin{FraseCelebre}
	\begin{Frase}
		" La fuerza no viene de la capacidad corporal, sino de la voluntad del alma". 
	\end{Frase}
	\begin{Fuente}
		Mahatma Gandhi
	\end{Fuente}
\end{FraseCelebre}


Introducci�n capitulo 3 COMPLETAR 
\\



%-------------------------------------------------------------------
\section{Flask}
%-------------------------------------------------------------------
\label{cap3:sec:Flask}
Flask es un ``micro'' Framework escrito en Python y concebido para facilitar el desarrollo de Aplicaciones Web y APIs.\\

La palabra micro hace referencia a que Flask �nicamente trae por defecto las herramientas necesarias para crear una aplicaci�n web b�sica, aunque si se necesitan a�adir nuevas funcionalidades hay un conjunto muy grande de extensiones (plugins) que se pueden instalar f�cilmente. Por ello, Flask es muy recomendable para el desarrollo de aplicaciones que no requieran muchas extensiones o que se necesiten implementar de una forma �gil y r�pida. Tambi�n es muy recomendable para implementar microservicios.\\

Algunas de las caracter�sticas de Flask por las que decidimos desarrollar nuestra API con este framework son las siguientes:\\

\newpage
\begin{itemize}
	
	\item Rapidez y facilidad en la instalaci�n y configuraci�n, a diferencia de otros frameworks como Django, que tiene una curva de aprendizaje mucho m�s baja.
	
	\item Es compatible con Python. Nuestra API est� implementada en dicho lenguaje.
	
	\item Incluye un servidor web de desarrollo. No se necesita una infraestructura con un servidor web para probar las aplicaciones, sino que de una manera sencilla se puede levantar un servidor web para ir viendo los resultados que se van obteniendo.
	
	\item Cuenta con depurador. Si tenemos alg�n error en el c�digo que se est� desarrollando, se puede depurar ese error y ver los valores de las variables.
	
	\item Flask es Open Source y est� amparado bajo una licencia BSD, que es la licencia utilizada para los sistemas operativos BSD (Berkeley Software Distribution), y tiene menos restricciones en comparaci�n con otras licencias como la GPL, estando muy cercana al dominio p�blico.
	
	\item Cuenta con una muy buena documentaci�n.
	
\end{itemize}

















