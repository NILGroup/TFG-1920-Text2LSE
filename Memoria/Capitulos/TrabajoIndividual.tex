%---------------------------------------------------------------------
%
%                          Trabajo individual
%
%---------------------------------------------------------------------

\chapter{Trabajo individual}


%-------------------------------------------------------------------
\section{Sara Vegas Ca�as}
%-------------------------------------------------------------------
\label{capTrabajoIndividual:sec:Sara}

Lo primero de todo fue una investigaci�n profunda de la discapacidad auditiva y de la Lengua de Signos Espa�ola por parte de los tres integrantes de este proyecto, de esta manera todos nos concienciamos de las dificultades que tienen todas las personas sordas para comunicarse en su d�a a d�a. Realic� una investigaci�n de las herramientas que tienen estas personas a su disposici�n, tanto posibles traductores a LSE, como bancos de v�deos e im�genes. Encontr� una variedad de aplicaciones, las cuales prob� para poder ver hasta donde llegaban las soluciones actuales para el problema de comunicaci�n de las personas sordas. \\

Cuando llegamos a la conclusi�n del proyecto que quer�amos hacer, empezamos con la investigaci�n de las posibles tecnolog�as que podr�amos usar en el proyecto. Mi investigaci�n en este tema se centr� en servicios web, sobre todo de tipo REST, Python y en Procesamiento del Lenguaje Natural, probando diferentes librer�as, como Spacy y NLTK, para poder elegir la que m�s se ajustase a nuestras necesidades.\\

Una vez que termin� la primera parte de investigaci�n, nos reunimos para juntar todas nuestras conclusiones y comenzamos a desarrollar la primera parte de la memoria con los conocimientos adquiridos en la investigaci�n previa.\\

A continuaci�n, comenzamos a crear una primera base de la aplicaci�n. Lo primero fue realizar la preparaci�n del entorno local en Debian 10, utilizando Nginx para alojar la aplicaci�n web, y Flask para ejecutar la API, preparando el modo debug para el posterior proceso de desarrollo de c�digo. Construimos una web b�sica, en la cual pod�amos escribir un texto y enviarlo a la API a trav�s de JavaScript, mediante jQuery y Ajax, y una primera estructura de la API REST en Python. En este apartado me encargue junto con mi compa�ero Alejandro del desarrollo de una funcionalidad javascript capaz de lanzar una petici�n a la API con el texto y que esta nos devolviese un video mp4.\\

En la fase posterior, me encargu� de corregir mis apartados correspondientes de la memoria, y de desarrollar los nuevos apartados de Python y metodolog�as utilizadas como Trello y Reuniones. Tambi�n me encargu� de estructurar la funci�n de la API que devuelve un video con varios signos y de desarrollar las llamadas a las funciones que devuelven diferente tipo de informaci�n en Json. Respecto a PLN, en un principio me centr� en desarrollar un algoritmo que recorra un �rbol de dependencias desde el verbo a partir de una frase.\\
