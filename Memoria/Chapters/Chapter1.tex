%---------------------------------------------------------------------
%
%                          Cap�tulo 1 Ingl�s
%
%---------------------------------------------------------------------

\begin{otherlanguage}{english}


\setcounter{chapter}{0}
\chapter{Introduction}

In section 1.1 of this chapter the motivation behind the implementation of this project will be explained, and in section 1.2 the objectives that we set ourselves at the beginning of the project. On the other hand, section 1.3 will detail the structure of this document.


%-------------------------------------------------------------------
\section{Motivation}
%-------------------------------------------------------------------
\label{cap1:sec:introduccion}

In the current society in which we live there are a large number of people with different disabilities: cognitive, physical, visual, hearing... In some cases, these disabilities make these people feel separated from the rest of society because they do not have complete access to certain fundamental services, such as public address announcements in public places such as airports or hospitals, which makes their day to day difficult. Thanks to technological advances, solutions are being developed to help these groups access different services and thus integrate them into society. An example is the applications based on pictograms (simple images that represent a concept), whose purpose is to help people with some kind of cognitive disability to communicate, which prevents them from communicating using natural language.\\

In Spain, 2.3\% of the total population (around one million people) suffers from some type of hearing disability. This group uses different communication methods, which depend both on the age at which they began to suffer deafness, and on their degree of hearing loss. People with hearing disabilities have the same needs to obtain information from the environment as the rest of the population, but despite having the same cognitive abilities, they face numerous communication barriers, which hinder their learning process and the ability to relate with their environment through listening and speaking. This is because on many occasions audio is the only channel to communicate information, be it in movies, news programs, daily conversations, educational activities, public address systems or videos.\\ 

The most widely used alternative to oral language among hearing impaired people is Sign Language (SL). The SL is not universal, but varies depending on the region, that is, there is no common sign language throughout the world, with each country having one or more. In Spain since 2007 there are two officially recognized languages: Spanish Sign Language (SSL) and Catalan Sign Language (CSL).\\

 
 Today not all audios are supplemented with the SL. This means that people with hearing disabilities do not finish being integrated into society. Deaf people are not in equal conditions when it comes to receiving information, since in most cases audio is the only channel to communicate information in movies, news programs, programs, notices in public places, etc. Of the cases there is the possibility of adding subtitles that accompany the audio, but it is not enough, since the subtitles are not as effective as the SL. The SL is ideographic (it allows representing common concepts with a single gesture), which makes it take less time to receive the information with the SL than with the subtitles. For all this, it is important to promote and facilitate the use of SL as an alternative language to the oral language.\\


Having a tool capable of translating a text into Sign Language would be of great help for people with hearing disabilities, since it would offer the possibility of introducing the SL to any audio that has subtitles, and even of directly translating any audio through the support of voice to text translation tools.





%-------------------------------------------------------------------
\section{Goals}
%-------------------------------------------------------------------
\label{cap1:sec:objetivo}
The main goal of the project is to create a free tool that is capable of translating in real time a text in natural language written in Spanish into Spanish Sign Language (SSL) in real time. The application created will allow the SSL to be incorporated into any audiovisual material in a simple way, and will also support people in the SSL learning process.\\

The application will be based on a Service Oriented Architecture (SOA), that is, it will have small functionalities implemented as web services, easily reusable by other programmers who wish to integrate the functionalities that we are going to develop into their applications. These microservices will receive a text in Spanish and return the translation to SSL in various formats (video and image), making use of Natural Language Processing (NLP) tools for this.\\


Regarding the academic objectives, with this project we intend to put into practice the knowledge acquired throughout the Degree in Computer Engineering, as well as expanding our knowledge and skills.\\


%-------------------------------------------------------------------
\section{Work Methodology}
%-------------------------------------------------------------------
\label{cap1:sec:estructura}

Chapter 2 discusses hearing impairment and the means of communication available to people with hearing impairment. There is also a deeper insight into Sign Language, specifically the Spanish Sign Language. This is complemented by an analysis of the tools that currently exist that help to communicate and learn the SSL. Chapter 3 presents the tools used to develop the application. Chapter 4 explains in detail the work methodology followed by the members of the group during the development of this project. Chapter 5 explains the work done in detail. Chapter 6 explains the evaluation and tests carried out on the developed application and details the results. In Chapter 7 each member of the group explains the individual work carried out throughout the project. Finally, Chapter 8 presents the conclusions and details some future work that could be carried out to improve the application.\\

\end{otherlanguage}
	
	


