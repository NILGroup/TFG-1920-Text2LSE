%---------------------------------------------------------------------
%
%                          Cap�tulo 1 Ingl�s
%
%---------------------------------------------------------------------


\chapter{Introduction}

The motivation behind the implementation of this project will be explained in section 1.1. The objectives that we set ourselves at the beginning of the project will be presented in Section 1.2. Finnaly, the structure of this document will be detailed in Section 1.3.


%-------------------------------------------------------------------
\section{Motivation}
%-------------------------------------------------------------------
\label{cap1:sec:introduccion}

In current society in which we live, there is a large number of people with different disabilities: cognitive, physical, visual, hearing... In some cases, these disabilities make these people feel separated from the rest of society because they do not have complete access to certain fundamental services, like public address announcements in public places, such as airports or hospitals, which makes difficult their daily life. Current technological advances have allowed the development of solutions that allow these groups to access different services and, therefore, integrate them into society. As an example, applications based on pictograms (simple images that represent a concept), help people with some level of cognitive disability (which does not allow them to communicate using natural language) to communicate.\\

In Spain, 2.3\% of the total population (around one million people) suffers from some type of hearing disability. This group uses different communication methods, which depend both on the age at which they become deaf, and on their degree of hearing loss. People with hearing disabilities have the same needs to obtain information as the rest of the population. But despite having the same cognitive abilities, they face numerous communication barriers, which hinder their learning process and the ability to relate to their environment through listening and speaking. This is because on many occasions audio is the only channel to communicating information, whether in films, news, everyday conversations, educational activities, public address systems or videos.\\ 

The alternative method to oral language most used among hearing impaired people is Sign Language (SL). The SL is not universal, it varies depending on the region, for example, there is no common wordwid sign language, but each country has one or more than one. In Spain, there are two officially recognized languages since 2007: Spanish Sign Language (SSL) and Catalan Sign Language (CSL).\\

 
 Nowadays not all audios are complemented by the SL. This means that hearing impaired people are not fully integrated into society. Deaf people are not on equal terms when it comes to receiving information, since audio is in most cases the main channel to communicate information in movies, news, programs, notices in public places, etc. In many cases subtitles can be added to accompany the audio, but that is not enough, since subtitles are not as effective as the SL. The SL is ideographic (it allows the representation of common concepts with a single gesture), which makes information be received faster through sign language. It is therefore important to promote and facilitate the use of SL as an alternative language to the oral language.\\


A tool capable of translating text into Sign Language would be of great help to hearing impaired people, as it would offer the possibility of introducing the SL to any subtitled audio, and even of directly translating any audio through the support of voice to text translation tools.





%-------------------------------------------------------------------
\section{Goals}
%-------------------------------------------------------------------
\label{cap1:sec:objetivo}
The main goal of this project is to create a free tool capable of translating a  natural language text written in Spanish into Spanish Sign Language (SSL) in real time. This application will allow the SSL to be embed on any audiovisual material in a simple way. It will also support people in the SSL learning process.\\

The application will be based on a Service Oriented Architecture (SOA): small functionalities will be implemented as web services, easily reusable by other programmers who wish to integrate the functionalities that we will develop into their applications. These microservices will receive a text in Spanish and return the SSL translation in different formats (video and image), using Natural Language Processing (NLP).\\


Regarding the academic objectives, our intention is to put into practice everything learned during the Degree in Computer Engineering, as well as to expand our knowledge and skills.\\


%-------------------------------------------------------------------
\section{Document structure}
%-------------------------------------------------------------------
\label{cap1:sec:estructura}

In Chapter 2 hearing impairment and the different ways of communication available for hearing impaired people are discussed. Also, a deeper insight into Sign Language, specifically Spanish Sign Language is provided. This is complemented by an analysis of the tools currently available to help communicate and learn about SSL. The tools used to develop the application are presented in Chapter 3. In Chapter 4, the working methodology followed by the members of the group during the development of this project is explained in detail. A detailed description of the undertaken work is provided in Chapter 5. The results of the evaluation and testing of the application are shown in Chapter 6. The individual work done throughout the project is explained in Chapter 7. Finally, in Chapter 8, conclusions are presented and some future work are suggested.\\


	
	


