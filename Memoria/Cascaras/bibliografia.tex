%---------------------------------------------------------------------
%
%                      configBibliografia.tex
%
%---------------------------------------------------------------------
%
% bibliografia.tex
% Copyright 2009 Marco Antonio Gomez-Martin, Pedro Pablo Gomez-Martin
%
% This file belongs to the TeXiS manual, a LaTeX template for writting
% Thesis and other documents. The complete last TeXiS package can
% be obtained from http://gaia.fdi.ucm.es/projects/texis/
%
% Although the TeXiS template itself is distributed under the 
% conditions of the LaTeX Project Public License
% (http://www.latex-project.org/lppl.txt), the manual content
% uses the CC-BY-SA license that stays that you are free:
%
%    - to share & to copy, distribute and transmit the work
%    - to remix and to adapt the work
%
% under the following conditions:
%
%    - Attribution: you must attribute the work in the manner
%      specified by the author or licensor (but not in any way that
%      suggests that they endorse you or your use of the work).
%    - Share Alike: if you alter, transform, or build upon this
%      work, you may distribute the resulting work only under the
%      same, similar or a compatible license.
%
% The complete license is available in
% http://creativecommons.org/licenses/by-sa/3.0/legalcode
%
%---------------------------------------------------------------------
%
% Fichero  que  configura  los  par�metros  de  la  generaci�n  de  la
% bibliograf�a.  Existen dos  par�metros configurables:  los ficheros
% .bib que se utilizan y la frase c�lebre que aparece justo antes de la
% primera referencia.
%
%---------------------------------------------------------------------


%%%%%%%%%%%%%%%%%%%%%%%%%%%%%%%%%%%%%%%%%%%%%%%%%%%%%%%%%%%%%%%%%%%%%%
% Definici�n de los ficheros .bib utilizados:
% \setBibFiles{<lista ficheros sin extension, separados por comas>}
% Nota:
% Es IMPORTANTE que los ficheros est�n en la misma l�nea que
% el comando \setBibFiles. Si se desea utilizar varias l�neas,
% terminarlas con una apertura de comentario.
%%%%%%%%%%%%%%%%%%%%%%%%%%%%%%%%%%%%%%%%%%%%%%%%%%%%%%%%%%%%%%%%%%%%%%
\setBibFiles{%
	otros%
}

%%%%%%%%%%%%%%%%%%%%%%%%%%%%%%%%%%%%%%%%%%%%%%%%%%%%%%%%%%%%%%%%%%%%%%
% Definici�n de la frase c�lebre para el cap�tulo de la
% bibliograf�a. Dentro normalmente se querr� hacer uso del entorno
% \begin{FraseCelebre}, que contendr� a su vez otros dos entornos,
% un \begin{Frase} y un \begin{Fuente}.
%
% Nota:
% Si no se quiere cita, se puede eliminar su definici�n (en la
% macro setCitaBibliografia{} ).
%%%%%%%%%%%%%%%%%%%%%%%%%%%%%%%%%%%%%%%%%%%%%%%%%%%%%%%%%%%%%%%%%%%%%%
%\setCitaBibliografia{
%	\begin{FraseCelebre}
%		\begin{Frase}
%			Y as�, del mucho leer y del poco dormir, se le sec� el celebro de
%			manera que vino a perder el juicio.
%		\end{Frase}
%		\begin{Fuente}
%			Miguel de Cervantes Saavedra
%		\end{Fuente}
%	\end{FraseCelebre}
%}

%%
%% Creamos la bibliografia
%%
\makeBib

%\bibliography{otros}
% Seleccionar Tesis.tex -> Herramientas -> Limpiar archivos auxiliares
%pdflatex Tesis.tex
%bibtex Tesis.aux
%bibtex Tesis.aux
%pdflatex Tesis.tex
%pdflatex Tesis.tex

% Variable local para emacs, para  que encuentre el fichero maestro de
% compilaci�n y funcionen mejor algunas teclas r�pidas de AucTeX

%%%
%%% Local Variables:
%%% mode: latex
%%% TeX-master: "../Tesis.tex"
%%% End:
