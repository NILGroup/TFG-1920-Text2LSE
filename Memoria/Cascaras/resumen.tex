%---------------------------------------------------------------------
%
%                      resumen.tex
%
%---------------------------------------------------------------------
%
% Contiene el cap�tulo del resumen.
%
% Se crea como un cap�tulo sin numeraci�n.
%
%---------------------------------------------------------------------

\chapter{Resumen}

En la sociedad actual en la que vivimos es imprecindible la comunicaci�n, ya que esta permite la integraci�n e interacci�n social. Actualmente muchas personas que sufren discapacidad auditiva se enfrentan a numerosas barreras comunicativas, y necesitan el apoyo de m�todos alternativos de comunicaci�n como la Lengua de Signos (LS) para integrarse completamente en la sociedad. Hoy en d�a casi ning�n material audiovisual cuenta con la LS integrada, lo que supone que muchas personas con discapacidad auditiva no tengan acceso a ciertos servicios fundamentales, como por ejemplo avisos por megafon�a en una estaci�n de tren o un hospital. En este TFG hemos desarrollado una herramienta gratuita que permite traducir un texto en lenguaje natural escrito a Lengua de Signos Espa�ola (LSE) en formato v�deo o imagen y de forma instant�nea, permitiendo as� incorporar la LS a cualquier material audiovisual, lo que contribuye favorablemente a la adapataci�n de este colectivo y su inclusi�n en la sociedad. A d�a de hoy no existe ninguna herramienta que sea capaz de traducir castellano a LSE en tiempo real, por lo que nuestro TFG puede ayudar a cubrir esa necesidad y mejorar la calidad de vida de las personas con discapacidad auditiva. La herramienta se ha desarrollado siguiendo un dise�o basado en componentes, por lo que otros desarrolladores pueden integrar nuestras funcionalidades en sus propias aplicaciones. Adem�s, se ha realizado una evaluaci�n de las traducciones proporcionadas por nuestra herramienta y, aunque hay mucho margen de mejora, los resultados han sido muy satisfactorios. Con el desarrollo de este TFg hemos establecido una buena base para la traducci�n a LS, ya que la aplicaci�n es capaz de traducir un n�mero considerable de oraciones, f�cilmente ampliable si se continua con el trabajo realizado.

\endinput
% Variable local para emacs, para  que encuentre el fichero maestro de
% compilaci�n y funcionen mejor algunas teclas r�pidas de AucTeX
%%%
%%% Local Variables:
%%% mode: latex
%%% TeX-master: "../Tesis.tex"
%%% End:
