%---------------------------------------------------------------------
%
%                      resumen.tex
%
%---------------------------------------------------------------------
%
% Contiene el cap�tulo del resumen.
%
% Se crea como un cap�tulo sin numeraci�n.
%
%---------------------------------------------------------------------

\chapter{Abstract}

In today's society, communication is essential, since it allows social integration and interaction. Nowadays, many people with hearing disabilities face numerous communication barriers and need the support of alternative communication methods, such as Sign Language (SL), to fully integrate into society. Today, most audiovisual material have not integrated SL, which means that many people with hearing disabilities do not have access to certain essential services, such as public address announcements at train stations or hospitals. In this project, we have developed a free tool that allows to instantly translate a Spanish text into Spanish Sign Language (SSL) in video or image format. Thus, Text2LSE will allow SL to be incorporated into any audiovisual material, which contributes favorably to the adaptation of this group and its inclusion into society. Nowadays, there is no tool that is capable of translating Spanish into SSL in real time, so our project can help cover that need and improve quality of life for people with hearing disabilities. The tool has been developed following a Service Oriented Architecture (SOA), so other developers will be able to integrate the functionalities of our application into their own applications. Additionally, an evaluation of the translations provided by our tool has been carried out. Although there is room for improvement, results have been very satisfactory. The development of this project allowed us to establish a good basis for the translation to SL, since the application is capable of translating a considerable number of sentences. Nevertheless, this work is easily expandable should it be continued.

\endinput
% Variable local para emacs, para  que encuentre el fichero maestro de
% compilaci�n y funcionen mejor algunas teclas r�pidas de AucTeX
%%%
%%% Local Variables:
%%% mode: latex
%%% TeX-master: "../Tesis.tex"
%%% End:
