%---------------------------------------------------------------------
%
%                      agradecimientos.tex
%
%---------------------------------------------------------------------
%
% agradecimientos.tex
% Copyright 2009 Marco Antonio Gomez-Martin, Pedro Pablo Gomez-Martin
%
% This file belongs to the TeXiS manual, a LaTeX template for writting
% Thesis and other documents. The complete last TeXiS package can
% be obtained from http://gaia.fdi.ucm.es/projects/texis/
%
% Although the TeXiS template itself is distributed under the 
% conditions of the LaTeX Project Public License
% (http://www.latex-project.org/lppl.txt), the manual content
% uses the CC-BY-SA license that stays that you are free:
%
%    - to share & to copy, distribute and transmit the work
%    - to remix and to adapt the work
%
% under the following conditions:
%
%    - Attribution: you must attribute the work in the manner
%      specified by the author or licensor (but not in any way that
%      suggests that they endorse you or your use of the work).
%    - Share Alike: if you alter, transform, or build upon this
%      work, you may distribute the resulting work only under the
%      same, similar or a compatible license.
%
% The complete license is available in
% http://creativecommons.org/licenses/by-sa/3.0/legalcode
%
%---------------------------------------------------------------------
%
% Contiene la p�gina de agradecimientos.
%
% Se crea como un cap�tulo sin numeraci�n.
%
%---------------------------------------------------------------------

\chapter{Agradecimientos}

\cabeceraEspecial{Agradecimientos}

\begin{FraseCelebre}
\begin{Frase}
A todos los que la presente vieren y entendieren.
\end{Frase}
\begin{Fuente}
Inicio de las Leyes Org�nicas. Juan Carlos I
\end{Fuente}
\end{FraseCelebre}

Groucho Marx dec�a que encontraba a la televisi�n muy educativa porque
cada vez que alguien la encend�a, �l se iba a otra habitaci�n a leer
un libro. Utilizando un esquema similar, nosotros queremos agradecer
al Word de Microsoft el habernos forzado a utilizar \LaTeX. Cualquiera
que haya intentado escribir un documento de m�s de 150 p�ginas con
esta aplicaci�n entender� a qu� nos referimos. Y lo decimos porque
nuestra andadura con \LaTeX\ comenz�, precisamente, despu�s de
escribir un documento de algo m�s de 200 p�ginas. Una vez terminado
decidimos que nunca m�s pasar�amos por ah�. Y entonces ca�mos en
\LaTeX.

Es muy posible que hub�eramos llegado al mismo sitio de todas formas,
ya que en el mundo acad�mico a la hora de escribir art�culos y
contribuciones a congresos lo m�s extendido es \LaTeX. Sin embargo,
tambi�n es cierto que cuando intentas escribir un documento grande
en \LaTeX\ por tu cuenta y riesgo sin un enlace del tipo ``\emph{Author
  instructions}'', se hace cuesta arriba, pues uno no sabe por donde
empezar.

Y ah� es donde debemos agradecer tanto a Pablo Gerv�s como a Miguel
Palomino su ayuda. El primero nos ofreci� el c�digo fuente de una
programaci�n docente que hab�a hecho unos a�os atr�s y que nos sirvi�
de inspiraci�n (por ejemplo, el fichero \texttt{guionado.tex} de
\texis\ tiene una estructura casi exacta a la suya e incluso puede
que el nombre sea el mismo). El segundo nos dej� husmear en el c�digo
fuente de su propia tesis donde, adem�s de otras cosas m�s
interesantes pero menos curiosas, descubrimos que a�n hay gente que
escribe los acentos espa�oles con el \verb+\'{\i}+.



\endinput
% Variable local para emacs, para  que encuentre el fichero maestro de
% compilaci�n y funcionen mejor algunas teclas r�pidas de AucTeX
%%%
%%% Local Variables:
%%% mode: latex
%%% TeX-master: "../Tesis.tex"
%%% End:
